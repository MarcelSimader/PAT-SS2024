% ViT included in content.tex
% Author: Marcel Simader (marcel0simader@gmail.com)
% Date: 04.11.2023
% License: See 'LICENSE'.

% ~~~~~~~~~~~~~~~~~~~~~~~~~~~~~~~~~~~~~~~~~~~~~~~~~~~~~~~~
% ~~~~~~~~~~~~~~~~~~~~ Document Class ~~~~~~~~~~~~~~~~~~~~
% ~~~~~~~~~~~~~~~~~~~~~~~~~~~~~~~~~~~~~~~~~~~~~~~~~~~~~~~~

\documentclass[utf8, aspectratio=169, english]{beamer}
% Define the aspect ratio of the slide layout:
%  * aspectratio=169  ... 16:9 aspect ratio
%  * aspectratio=43   ... 4:3 aspect ratio
%  * aspectratio=1610 ... 16:10 aspect ratio
% Define document languages:
%  * ngerman ... German
%  * english ... English
%  * ...
% Switch to handout mode:
%  * handout ... A compact mode that allows you to remove animation and skips slides for
%                efficient printing.
% Other options:
%  * utf8 ... Treat input files as UTF-8 encoded. Make sure to always provide that option
%             when you use pdfLaTeX so that pdfLaTeX knows how to read and interpret
%             characters this source file.

\setbeamercovered{transparent}

% ~~~~~~~~~~~~~~~~~~~~~~~~~~~~~~~~~~~~~~~~~~~~~~~
% ~~~~~~~~~~~~~~~~~~~~ Theme ~~~~~~~~~~~~~~~~~~~~
% ~~~~~~~~~~~~~~~~~~~~~~~~~~~~~~~~~~~~~~~~~~~~~~~

%\usepackage[TNF,nosectionpage]{style/jku}
\usepackage[%
    darkmode, fancyfonts, mathastext,%
    framenumber, totalframenumber,%
    TNF, logopath={./style/logos},fontpath={./style/fonts}]{style/beamerthemejku}
% Color scheme selection options:
%  * JKU  ... Use JKU (gray) color scheme (this is the default if no scheme is selected).
%  * BUS  ... Use Business School color scheme.
%  * LIT  ... Use Linz Institute of Technology color scheme.
%  * MED  ... Use MED faculty color scheme.
%  * RE   ... Use RE faculty color scheme.
%  * SOE  ... Use School of Education color scheme.
%  * SOWI ... Use SOWI faculty color scheme.
%  * TNF  ... Use TNF faculty color scheme.
% Color mode selection options:
%  * darkmode ... Use dark color mode (where title and logo frames have a dark background).
% Frame numbering options:
%  * framenumber         ... Insert frame number into the frame footer.
%  * totalframenumber    ... Insert frame number and total frame number into the frame footer
%                            (only frames in the main part are counted).
%  * appendixframenumber ... Similar to `totalframenumber', but count the overall total frame
%                            number of main part and appendix.
% Note that combining `totalframenumber' and `appendixframenumber' options will show the total
% number of frames for the main part on frames in the main part and the overal total number of
% frames for frames in the appendix.
% Sectioning options:
%  * nosectionpage       ... Supress section frames (see \section{<title>} command).
%  * nosubsectionpage    ... Supress subsection frames (see \subsection{<title>} command).
%  * nosubsubsectionpage ... Supress subsubsection frames (see \subsubsection{<title>} command).
%  * partpage            ... Insert part frames (see \part{<title>} command).
% Space-efficient monospace font options (requires XeTeX):
%  * compactmono   ... Use condensed fixed-width font everywhere.
%  * nocompactverb ... Do not use condensed fixed-width font for verbatim and listings.
% Style-breaking options:
%  * nojkufooter    ... Do not insert JKU/partner logos into the frame footer.
%  * nofooter       ... Do not display a frame footer.
%  * noimprint      ... Do not insert imprint on title pages.
%  * nojkulogo      ... Do not insert JKU & K logos on title pages and in frame footers.
%  * frametitlecaps ... Set frame titles in capital letters (like in eariler theme versions).
%  * nofancyfonts   ... Do not use custom TTF fonts with XeTeX / supress pdfLaTeX warning.
%  * mac            ... Use adapted color palette for screen display on Mac.
%  * legacyitemizestyle ... Use old bullet style in itemization.
% Experimental options:
%  * mathastext ... Use standard document fonts (and default to sans-serif font) in math mode
% Advanced options:
%  * nooptpackages     ... Do not load additional convenience packages (which are only there
%                          to provide interoperability to the behavior of previous versions of
%                          this theme but are not actually required for the current version).
%  * logopath={<path>} ... Set the path where the theme can find its own logo resources. This
%                          should typically be a relative path and the default is `./logos'.
%  * fontpath={<path>} ... Set the path where the theme can find its own font resources. This
%                          should typically be a relative path and the default is `./fonts'.

% ~~~~~~~~~~~~~~~~~~~~~~~~~~~~~~~~~~~~~~~~~~~~~~~~~~~~~~~~~~~
% ~~~~~~~~~~~~~~~~~~~~ Included Packages ~~~~~~~~~~~~~~~~~~~~
% ~~~~~~~~~~~~~~~~~~~~~~~~~~~~~~~~~~~~~~~~~~~~~~~~~~~~~~~~~~~

% Default stuff...
\usepackage{xspace}
\usepackage{multicol}
\usepackage{hyperref}
\usepackage{csquotes}
% \usepackage{pifont}
% \usepackage{bold-extra}
\usepackage{xcolor}
% \usepackage{caption}
\usepackage{acronym}
% \usepackage[acronym]{glossaries}
\usepackage{environ}

% Maths stuff...
\usepackage{amssymb}
\usepackage{mathtools}
\usepackage{amsthm}

% Table stuff...
\usepackage{booktabs}
% \usepackage{tabularx}
% \usepackage{threeparttable}

% Tikz stuff...
% \usepackage{tikz}
% \usetikzlibrary{shapes.geometric}
% \usetikzlibrary{positioning}

% Float figure stuff...
\usepackage{listings}
% \usepackage{wrapfig}
% \setlength{\columnsep}{20pt}

% BibLaTeX stuff...
\usepackage[%
    backend=biber,sortcites=true,%
    style=style/ACM-Reference-Format]{biblatex}
\preto{\bibsetup}{\providecommand*{\insertbiblabel}{}}
\DeclareFieldFormat*{title}{#1}
\DeclareFieldFormat*{booktitle}{#1}
\DeclareFieldFormat*{journaltitle}{#1}
\setcounter{biburlnumpenalty}{100}
\setcounter{biburllcpenalty}{100}
\setcounter{biburlucpenalty}{100}
\addbibresource{literature.bib}

% ~~~~~~~~~~~~~~~~~~~~~~~~~~~~~~~~~~~~~~~~~~~~~~~~
% ~~~~~~~~~~~~~~~~~~~~ Macros ~~~~~~~~~~~~~~~~~~~~
% ~~~~~~~~~~~~~~~~~~~~~~~~~~~~~~~~~~~~~~~~~~~~~~~~

\def\P(#1){\ensuremath{P\mkern-0.5mu\left( #1 \right)}}

\NewEnviron{m-question}{%
    \begingroup
    \setbeamercolor{block title}{fg=white,bg=jkuPurple}
    \begin{block}{Research Question}
        \begin{displayquote}
            \large\vspace*{0.25\baselineskip}%
            \openautoquote{\BODY}\closeautoquote%
            \vspace*{0.2\baselineskip}
        \end{displayquote}
    \end{block}
    \endgroup}

\NewEnviron{m-problem}{%
    \begingroup
    \setbeamercolor{block title}{fg=white,bg=jkuRed}
    \begin{block}{Problem}
        \large\vspace*{0.25\baselineskip}{\BODY}\vspace*{0.2\baselineskip}
    \end{block}
    \endgroup}
