% ViT 1x make
% Authors: Marcel Simader (marcel.simader@jku.at)
%          Melissa Frischherz
% Date: 20.03.2024
\documentclass[conference]{IEEEtran}

% Preamble {{{
% ~~~~~~~~~~~~~~~~~~~~~~~~~~~~~~~~~~~~~~~~~~~~~~~~~~
% ~~~~~~~~~~~~~~~~~~~~ Preamble ~~~~~~~~~~~~~~~~~~~~
% ~~~~~~~~~~~~~~~~~~~~~~~~~~~~~~~~~~~~~~~~~~~~~~~~~~

\usepackage{amsmath,amssymb,amsfonts}
\usepackage{algorithmic}
\usepackage{graphicx}
\usepackage{textcomp}
\usepackage[dvipsnames]{xcolor}

\usepackage{cite}
\def\BibTeX{{\rm B\kern-.05em{\sc i\kern-.025em b}\kern-.08em
    T\kern-.1667em\lower.7ex\hbox{E}\kern-.125emX}}
\bibliographystyle{./IEEEtran}

% ~~~~~~~~~~~~~~~~~~~~ Custom Packages ~~~~~~~~~~~~~~~~~~~~

\usepackage{xspace}

% ~~~~~~~~~~~~~~~~~~~~ Custom Commands ~~~~~~~~~~~~~~~~~~~~

\newcommand{\TODO}[1]{\textbf{\textcolor{Bittersweet}{#1}}\xspace}
\newcommand{\TODOM}[1]{\TODO{\emph{TODO}: #1}\xspace}
\newcommand{\TODOB}{\TODO{\emph{TODO}}\xspace}

% ~~~~~~~~~~~~~~~~~~~~ Custom Settings ~~~~~~~~~~~~~~~~~~~~

\graphicspath{./figures}

% }}}

% Document {{{
% % ~~~~~~~~~~~~~~~~~~~~~~~~~~~~~~~~~~~~~~~~~~~~~~~~~~
% % ~~~~~~~~~~~~~~~~~~~~ Document ~~~~~~~~~~~~~~~~~~~~
% % ~~~~~~~~~~~~~~~~~~~~~~~~~~~~~~~~~~~~~~~~~~~~~~~~~~

\begin{document}

% Title {{{
% ~~~~~~~~~~~~~~~~~~~~ Title ~~~~~~~~~~~~~~~~~~~~
\title{Exploring Existing Machine Learning Approaches Discerning the Quality of Source
Code Identifiers}

\author{\IEEEauthorblockN{Marcel Simader}
\IEEEauthorblockA{\textit{AI for Software Engineering, Topic 3} \\
\textit{k11823075 \textfractionsolidus{} SKZ 521}\\
marcel.simader@jku.at}
\and
\IEEEauthorblockN{Melissa Frischherz}
\IEEEauthorblockA{\textit{AI for Software Engineering, Topic 3} \\
\textit{k12011649 \textfractionsolidus{} SKZ 521}\\
melissa.frischherz@gmail.com}
}

\maketitle
% }}}

% Abstract {{{
% ~~~~~~~~~~~~~~~~~~~~ Abstract ~~~~~~~~~~~~~~~~~~~~
\begin{abstract}
    The quality of source code identifiers has a direct, and measurable impact on program
    comprehension, which indirectly influences validity, maintainability, and security of
    software. Exploiting natural language processing (NLP) and machine learning models,
    such as large language models (LLM), could prove to be a powerful addition to the code
    analyst's toolkit in the future. In this paper, we will perform a small-scale survey
    of existing approaches attempting to determine the quality of identifiers. Such a
    measure can be used to evaluate naming conventions in big code bases, suggest new
    context-aware names to developers on-the-fly, or even find semantic relationships
    between identifiers across languages and programs.
\end{abstract}

\begin{IEEEkeywords}
Machine Learning, Natural Language Processing, Code design, Maintainability, Software
Quality/SQA.
\end{IEEEkeywords}
% }}}

% Introduction {{{
% ~~~~~~~~~~~~~~~~~~~~ Introduction ~~~~~~~~~~~~~~~~~~~~
\section{Introduction}
\label{sec:Introduction}

\TODOM{Explain kinds of identifiers, and why they matter. Rich information content of
atomic parts of a programming language's AST.}

\TODOM{State (and show evidence for) importance of good naming for quality assurance of
software, maintainability, etc.}

% }}}

% Background {{{
% ~~~~~~~~~~~~~~~~~~~~ Background ~~~~~~~~~~~~~~~~~~~~
\section{Background}
\label{sec:Background}

% Machine Learning {{{
\subsection{Machine Learning}
\label{ssec:Machine-Learning}

\TODOM{Brief discussions of supervised/unsupervised techniques, neural networks, deep
learning, and RNNs.}

% }}}

% Machine Learning {{{
\subsection{Natural Language Processing}
\label{ssec:Natural-Language-Processing}
% }}}

\TODOM{Brief discussions of $n$-grams, grammar, and the attention mechanism.}

% }}}

% Survey {{{
% ~~~~~~~~~~~~~~~~~~~~ Survey ~~~~~~~~~~~~~~~~~~~~
\section{Survey}
\label{sec:Survey}

% Traditional Heuristic Approaches {{{
\subsection{Traditional Heuristic Approaches}
\label{ssec:Traditional-Heuristic-Approaches}
\TODOM{``Intelligent'' refactoring and renaming, e.g.\@ in IntelliJ IDEA.}
% }}}

% RNNs and LSTMs {{{
\subsection{RNNs and LSTMs}
\label{ssec:RNNs and LSTMs}
\TODOM{Literature like the LSTM approach of ``A Neural Model for Method Name Generation
from Functional Description'' by Sa Gao et al.\cite{Gao2019IdentGen}}
% }}}

% LLMs and Beyond {{{
\subsection{LLMs and Beyond }
\label{ssec:LLMs-and-Beyond }
\TODOM{Literature like the LLM approach of ``How Well Can Masked Langauge Models Spot
Identifiers That Violate Naming Guidelines?'' by Johannes Villmow et
al.\cite{Villmow2023Violations}}
% }}}

% }}}

% Discussion and Future Work {{{
% ~~~~~~~~~~~~~~~~~~~~ Discussion and Future Work ~~~~~~~~~~~~~~~~~~~~
\section{Discussion and Future Work}
\label{sec:Discussion-and-Future-Work}

\TODOM{Potential for fully-integrated tools, or CI pipeline utilities. Application of
cutting-edge large language models. Surprising lack of experiments with GPTs?}
% }}}

% Conclusion {{{
% ~~~~~~~~~~~~~~~~~~~~ Conclusion ~~~~~~~~~~~~~~~~~~~~
\section{Conclusion}
\label{sec:Conclusion}

\TODOB
% }}}

% References {{{
% ~~~~~~~~~~~~~~~~~~~~ References ~~~~~~~~~~~~~~~~~~~~

% WARNING: This is a temporary macro to show references before we cite them in the text.
% TODO(Marcel): REMOVE after outline completion
\nocite{*}

\bibliography{IEEEfull,literature}
% }}}

\onecolumn
\appendix{Additional Notes}

\begin{verbatim}
    Formulation of Topic Keywords/Key Phrases
    ------------------------------------------------------------------------------------

    Topic is `AI in Software Engineering`:
        - Natural Language Processing (NLP)
        - Large Language Models
        - Recurrent Neural Networks (RNN) and Encoder-Decoder models
        - Code quality measurement, improvement and assurance
        - Handling of identifiers in source code -- atomic units of knowledge, making up
          most of all source code
        - Understanding identifier in a functional sense (what do they mean to say?) and
          making (cross-language) connections between them (are they about the same
          thing?)

    Research Question
    ------------------------------------------------------------------------------------

    We answer the following questions[^1] in a maximum of two sentences, optimally one.
    Statements here are not cited, as they are not part of the academic text. Beware of
    plagiarism -- do not use anything from here directly.

    Introduction:
    > Identifiers make up about 70% of all source code, and hence encode a lot of
    > information in a natural language format (with additional syntactic rules.) Code
    > quality and comprehension might be dramatically improved by considering this
    > information.

    Problem Statement:
    > Since identifiers are natural language, this information can be hard to make sense
    > of with automated tooling for code analysis, improvement, automatic completion, or
    > quality assurance.

    State of the Art (in Literature):
    > There exist NLP models, mostly based on deep learning but also some statistical
    > analysis, which can predict identifiers to aid developers, measure the quality of
    > existing identifiers, and establish semantic relationships between identifiers,
    > within or across languages.

    Our approach:
    > We collect a list of literature presenting state-of-the-art machine learning
    > techniques to draw broader conclusions from: Effectiveness of current approaches,
    > categories of implementation, and possible pathways for future research.

    Implementation:
    > [ Not applicable? Err... we write the paper? ]

    Results:
    > Ideally, we present the reader with information which helps them understand the
    > landscape of identifier-specific software engineering tools, and inspire them
    > to find gaps in knowledge, or improve on existing ideas.

    [^1]: Writing template, courtesy of Steve Eastbrook.
\end{verbatim}

\end{document}
% }}}

% vim: foldmethod=marker
