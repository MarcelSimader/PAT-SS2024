% ViT 2x on-write make
% Authors: Marcel Simader (marcel.simader@jku.at)
%          Melissa Frischherz (melissa.frischherz@gmail.com)
% Date: 20.03.2024
\documentclass[conference]{IEEEtran}

% Preamble {{{
% ~~~~~~~~~~~~~~~~~~~~~~~~~~~~~~~~~~~~~~~~~~~~~~~~~~
% ~~~~~~~~~~~~~~~~~~~~ Preamble ~~~~~~~~~~~~~~~~~~~~
% ~~~~~~~~~~~~~~~~~~~~~~~~~~~~~~~~~~~~~~~~~~~~~~~~~~

\usepackage{amsmath,amssymb,amsfonts}
\usepackage{algorithmic}
\usepackage{graphicx}
\usepackage{textcomp}
\usepackage{booktabs}
\usepackage[dvipsnames]{xcolor}
\usepackage{cite}

% ~~~~~~~~~~~~~~~~~~~~ Custom Packages ~~~~~~~~~~~~~~~~~~~~

\usepackage{xspace}
\usepackage{tikz}
\usepackage{pgfplots}
\usepackage[nohyperlinks, printonlyused]{acronym}
\usepackage{csquotes}

% ~~~~~~~~~~~~~~~~~~~~ Custom Commands ~~~~~~~~~~~~~~~~~~~~

\newcommand{\TODO}[1]{\textbf{\textcolor{Bittersweet}{#1}}\xspace}
\newcommand{\TODOM}[1]{\TODO{\emph{TODO}: #1}\xspace}
\newcommand{\TODOB}{\TODO{\emph{TODO}}\xspace}
\def\P(#1){\ensuremath{P\mkern-0.5mu\left( #1 \right)}}

% ~~~~~~~~~~~~~~~~~~~~ Custom Settings ~~~~~~~~~~~~~~~~~~~~

\bibliographystyle{./IEEEtran}
\graphicspath{./figures}
\pgfplotsset{compat=1.18}

% }}}

% Document {{{
% ~~~~~~~~~~~~~~~~~~~~~~~~~~~~~~~~~~~~~~~~~~~~~~~~~~
% ~~~~~~~~~~~~~~~~~~~~ Document ~~~~~~~~~~~~~~~~~~~~
% ~~~~~~~~~~~~~~~~~~~~~~~~~~~~~~~~~~~~~~~~~~~~~~~~~~

\begin{document}

% ~~~~~~~~~~~~~~~~~~~~ Acronyms ~~~~~~~~~~~~~~~~~~~~

\newacro{AI}[AI]{Artificial Intelligence}
\newacro{ML}[ML]{Machine Learning}
\newacro{NLP}[NLP]{Natural Language Processing}
\newacro{LM}[LM]{Language Model}
\newacro{LLM}[LLM]{Large Language Model}
\newacro{NN}[NN]{Neural Network}
\newacro{DNN}[DNN]{Deep Neural Network}
\newacro{RNN}[RNN]{Recurrent Neural Network}
\newacro{LSTM}[LSTM]{Long Short-Term Memory}
\newacroplural{LSTM}[LSTMs]{Long Short-Term Memory Models}
\newacro{BPE}[BPE]{Byte Pair Encoding}
\newacro{OOV}[OOV]{Out of Vocabulary}
\newacro{LBL}[LBL]{Logbilinear}
\newacro{GPT}[GPT]{Generative Pre-Trained Transformer}

% Title {{{
% ~~~~~~~~~~~~~~~~~~~~ Title ~~~~~~~~~~~~~~~~~~~~
\title{Exploring Existing Machine Learning Approaches Discerning the Quality of Source
Code Identifiers}

\author{\IEEEauthorblockN{Marcel Simader}
\IEEEauthorblockA{\textit{\acs{AI} for Software Engineering, Topic 3} \\
\textit{k11823075 \textfractionsolidus{} SKZ 521}\\
marcel.simader@jku.at}
\and
\IEEEauthorblockN{Melissa Frischherz}
\IEEEauthorblockA{\textit{\acs{AI} for Software Engineering, Topic 3} \\
\textit{k12011649 \textfractionsolidus{} SKZ 521}\\
melissa.frischherz@gmail.com}
}
\acused{AI}

\maketitle
% }}}

% Abstract {{{
% ~~~~~~~~~~~~~~~~~~~~ Abstract ~~~~~~~~~~~~~~~~~~~~
\begin{abstract}

    The quality of source code identifiers has a direct, and measurable impact on program
    comprehension, indirectly influencing validity, maintainability, and security of
    software. Using \acp{LM} to extract semantic information from an identifier is an
    important but challenging task, that is fundamental to assessing its quality. This
    paper proposes a small-scale survey of existing approaches to quantify how
    ``suitable'' a given identifier is in some program context. Such a measure facilitates
    the evaluation of naming conventions in big code bases, suggestions of context-aware
    variable, method, or type names, or even finding semantic relationships between
    identifiers across languages and programs. By leveraging promising advancements in
    \ac{NLP} and machine learning, particularly the recent \acp{LLM}, to analyze the
    information found in identifiers, we could greatly improve the code analyst's toolkit.

\end{abstract}
\acresetall

\begin{IEEEkeywords}
Machine Learning, Natural Language Processing, Code design, Maintainability, Software
Quality/SQA.
\end{IEEEkeywords}
% }}}

% Introduction {{{
% ~~~~~~~~~~~~~~~~~~~~ Introduction ~~~~~~~~~~~~~~~~~~~~
\section{Introduction}
\label{sec:Introduction}

Identifiers are the smallest unit of semantic information in program source code, yet they
make up the majority of it~\cite{Butler2010Empirical}. This makes them both important to
developers, who need them to comprehend software systems,  and practically indispensable
to nearly all analytical tools~\cite{Gao2019IdentGen}. An empirical study published by
Butler et~al.\@ shows that making poor choices for method, class, and type names
correlates with less readable, maintainable, and overall worse quality
code~\cite{Butler2010Empirical}. Recently, \ac{ML} and \ac{NLP} approaches have lead to
the development of useful code analysis tools, such as the ``Anti-Pattern'' detector by
Arnaoudova et al.\@~\cite{Arnaoudova2013LAs}. At the heart of such modern tools lies the
extraction of meaning behind identifiers chosen by imperfect developers, who may introduce
questionable naming schemes, spelling errors, or semantically faulty names. The
information found in good identifiers, on the other hand, can be used to gather a rich
understanding of a program's structure and intended function.

In the following sections, we present a survey of various papers with a focus on
identifiers in the context of software engineering and \ac{AI}. We pay special attention
to papers that attempt to qualitatively assess identifier names, which opens the path to
generating context-dependant identifier suggestions, or finding failures to adhere to
(implicitly) established naming conventions, for example. With this study, we aim to
provide an entry point into this area of research in the hopes of sparking new ideas,
highlighting current gaps in knowledge, and inspiring connections to be made across
different fields of research. At the end of this paper, the reader should have a baseline
understanding of relevant techniques and technologies, from which new research could be
conducted.

Section~\ref{sec:Background} will give some background on the theoretical and technical
frameworks, section~\ref{sec:Methodology} will discuss how we searched for, and selected
papers, section~\ref{sec:Survey} will present the findings of the survey, and finally
sections~\ref{sec:Discussion-and-Future-Work}~and~\ref{sec:Conclusion} will conclude it
with a discussion and ideas for future work.

% }}}

% Background {{{
% ~~~~~~~~~~~~~~~~~~~~ Background ~~~~~~~~~~~~~~~~~~~~
\section{Background}
\label{sec:Background}

% Natural Language Processing {{{
\subsection{\acl{NLP}}
\label{ssec:Natural-Language-Processing}

\ac{NLP} has been an active area of research since the nineteen nineties, and was proposed
as possible solution to machine translation as early as the fifties. The foundation for
the grammar model used in most \ac{NLP} today is the modelling of probabilities of a word
$s_i$ occurring given the context of the previous words in some sentence $s_1 s_2 \ldots
s_i$. In essence, the problem is to find a probability distribution over these sentences,
so that $\P(s_1 s_2 \ldots s_i) = \P(s_1) \P(s_2 \mid s_1) \cdots \P(s_i \mid s_1 \ldots
s_{i-1})$ aligns with the behavior we want from the system, e.g.\@ summarizing a text.
Historically, resources were limited, and even in ignoring the exponential nature of the
problem, computing a history of hundreds of conditional probabilities was simply not
feasible. This is why the $n$-gram model was developed, which ``chops off the tail end''
of the context, thereby reducing it to $n - 1$ conditional
probabilities~\cite{Brown1990StatMT}.

As Allamanis et al.\@~\cite{Allamanis2015Suggesting} note, an issue with the $n$-gram
model for modelling identifiers is that they often contain neologisms. That is, words that
do not appear anywhere in the training set, also called the ``vocabulary'' of the model.
This is self-evident when looking at an example of a getter-method name \verb-getState-,
which could also be written as \verb-state-, \verb-retrieveState-,  \verb-getsState-, etc.
An identifier model needs to be capable of handling \emph{out-of-vocabulary words}: This
is known as the \ac{OOV} problem~\cite{Shi2022Splitting}.

% }}}

% Machine Learning {{{
\subsection{\acl{ML}}
\label{ssec:Machine-Learning}

Classical statistical techniques paved the way for neural networks in \ac{NLP} in the 21st
century. The basis of a \ac{NN} is the neuron, which is found in layers, and sums up
inputs and applies some activation function to get an output. The intent with neurons is
to model human brains to the best of our current knowledge. Deep learning is the method
used to train \acp{DNN}, which typically contain more than four internal layers of
neurons, also called ``hidden layers''~\cite{Sze2017DNNs}. Naturally, recognition of
language is well-suited for \acp{DNN}, as they can be trained to provide the probabilities
discussed in subsection~\ref{ssec:Natural-Language-Processing}.

\acp{DNN} in their basic form are still incapable of handling arbitrary context lengths,
not unlike the $n$-gram. A type of \ac{DNN}, the \ac{RNN} was developed to handle
sequential data of arbitrary length. It achieves this by feeding its own hidden layer
output back into itself, mimicking a basic form of memory~\cite{Gao2019IdentGen}. One
variant of the \ac{RNN}, the \ac{LSTM} model, takes advantage of restrictions on the
information flow through itself to improve retention~\cite{Hochreiter97LSTM}.

There are many more variants of these models, arranged in different architectures. The
encoder-decoder \ac{RNN}, and the transformer are two such examples. The transformer
appears to be the best neural model for language at the moment, and is also represented in
our study. Instead of using recurrence, it relies on the attention mechanism: This allows
the model to weigh the importance of different parts of a sentence in relation to the
currently computed probabilities. The breakthrough achieved by Vaswani et
al.\@~\cite{Vaswani2017Transformer} was to utilize self-attention, a mechanism by which
the model can also weigh the importance of weights within itself. In praxis, this means
that the model learns better approximations of how to \emph{derive} the semantics of a
sentence, not just which features to focus on.

% }}}

% }}}

% Methodology {{{
% ~~~~~~~~~~~~~~~~~~~~~~~~~~~~~~~~~~~~~~~~~~~~~~~~~~~~~
% ~~~~~~~~~~~~~~~~~~~~ Methodology ~~~~~~~~~~~~~~~~~~~~
% ~~~~~~~~~~~~~~~~~~~~~~~~~~~~~~~~~~~~~~~~~~~~~~~~~~~~~
\section{Methodology}
\label{sec:Methodology}

We started by establishing an initial overview of the topic landscape, which led us to a
more systematic search of papers pertaining to the ideas discussed in
Section~\ref{sec:Introduction}. This was done by crafting search strings with relevant
keywords for IEEEXplore\footnote{https://ieeexplore.ieee.org} and Google
Scholar\footnote{https://scholar.google.com}. The search queries we used for both services
are as follows:

\begin{description}[\IEEEsetlabelwidth{Google Scholar:}]
    \item[IEEEXplore:]

    \enquote{(AI OR LLM OR Language Model OR NLP) AND Identifiers AND %
             (Completion OR Analysis OR Suggestion)}

    \item[Google Scholar:]

    \enquote{Large Language Models for measuring quality of identifiers in source code}
\end{description}

Of the 190 papers the IEEEXplore query yielded, we hand selected 30 to include in broader
analysis. For the Google Scholar search query, we hand selected 4 papers directly. In
total, we used 34 studies to establish the landscape, and looked further into 7 of them.
Any other papers cited in the work were used for supplemental information.

% }}}

% Survey {{{
% ~~~~~~~~~~~~~~~~~~~~ Survey ~~~~~~~~~~~~~~~~~~~~
\section{Survey}
\label{sec:Survey}

Research into the significance of identifiers in software engineering has been ongoing
since at least 1999, which is the oldest paper in our survey~\cite{Antoniol1999OO}, but
has only recently gained in popularity. A histogram of the years of publication for our
survey papers can be seen in figure~\ref{fig:Histogram-of-Survey-Paper-Publications}. We
suspect this is due to the major improvements in \ac{NLP} around the time Google released
its paper on the transformer in 2017~\cite{Vaswani2017Transformer}, Peters et al.\@
presented ELMo in the following year~\cite{Peters2018DeepCW}, and finally, BERT was
proposed by Devlin et al.\@ in 2019~\cite{Devlin2019BERT}. This aligns with the big spike
in publications that same year.

\begin{figure}[h!]
    \begin{tikzpicture}
        \begin{axis}[%
            ybar, bar width=5pt, ylabel={\# Publications}, xlabel={Year},
            enlargelimits=0.03,
            symbolic x coords={1999, 2000, 2001, 2002, 2003, 2004, 2005, 2006, 2007,%
                               2008, 2009, 2010, 2011, 2012, 2013, 2014, 2015,%
                               2016, 2017, 2018, 2019, 2020, 2021, 2022, 2023,%
                               2024
                               }]
             \addplot+[y filter/.expression={y==0 ? nan : y}, line width=0.2pt]%
                table [col sep=comma, header=true]%
                {./data/ext-pubdates.csv};
        \end{axis}
    \end{tikzpicture}
    \caption{Histogram of Survey Paper Publications}
    \label{fig:Histogram-of-Survey-Paper-Publications}
\end{figure}

We have divided the survey into three subsections, one on \enquote{traditional}
algorithmic approaches, one on approaches using \acp{RNN} and \acp{LSTM}, and one on very
recent advancements like transformers, and \acp{LLM}.

% Algorithmic Approaches {{{
\subsection{Algorithmic Approaches}
\label{ssec:Algorithmic-Approaches}

The simplest approach to quantifying the correct naming of identifiers in our survey is
found in the paper by Charitsis et al.\@~\cite{Charitsis2021Assessing}, which focuses on
the skills of computer science students. A reduced syntax backed by the Java programming
language named Karel is used along with student submissions to create a simple,
manually-annotated dataset of function name quality. Students submitted their source code
solutions for a problem statement in Karel. The submitted code was compiled in-memory and
instrumented, wherein the program state was compared before and after every function
execution to produce a pairwise similarity score of each student function. This similarity
score, along with the manually annotated quality is used to train a discriminator model
producing scores on unseen identifiers. Charitsis et al.\@ report classifier accuracies
between 89.36\% and 76.16\%, depending on the problem hardness.

We also examined a paper which aims to tackle the \ac{OOV} problem (see
subsection~\ref{ssec:Natural-Language-Processing}) by means of identifier
splitting~\cite{Shi2022Splitting}. Shi et al.\@ show that by modifying the \ac{BPE}
algorithm, which was originally developed for compression but later adopted by Karampatsis
et al.\@~\cite{Karampatsis2020BPE}, the performance of language models can be improved
slightly. In their study, they argue that the semantics of identifiers, e.g.\@
\verb-getHeader-, is lost in the \ac{BPE} scheme. A better approach would be to split the
identifier by certain heuristics -- the Ronin approach in this case -- and turn the
previous example into \verb-get- and \verb-Header-. Splitting identifiers with high
accuracy is crucial in this domain, which makes solutions to this problem particularly
sought after: Allamanis et al.\@ present another approach in which they split a token into
subtokens~\cite{Allamanis2015Suggesting}.

The final publication foregoing neural networks in our survey is the extensive article on
identifier renaming and prediction suggestion by Zhang et
al.~\cite{Zhang2023RenamingPrediction}. This is the only study we analyzed which uses a
decision-tree based model. More accurately, it uses several decision trees in an
arrangement called a \enquote{Random Forest}, from which an aggregate result is drawn. A
particularly interesting facet of this work is an emphasis on the fact that similar
identifiers usually change in similar ways, which lead the authors to utilize file
histories to aid in improving accuracy.

% }}}

% DNNs and RNNs {{{
\subsection{\acsp{DNN} and \acp{RNN}}
\label{ssec:DNNs-RNNs-and-LSTMs}

\acp{DNN} seem to be by far the most common approach in modern papers on identifier
quality assessment. Allamanis et al.\@ present a variation on the \ac{LBL} \ac{LM}, the
logbilinear context model~\cite{Allamanis2015Suggesting}. This neural network extends the
long-distance context of the \ac{LBL}~\ac{LM} to encompass what is called the local and
global context. This allows the model to more efficiently learn from an identifier's
surroundings at \emph{every occurrence} of said identifier, along with a feature vector of
general properties (e.g.\@ \enquote{constant value}, \enquote{integer type}). This, in
combination with the exploitation of the semantic substructures of identifiers mentioned
in the previous subsection, allows the model to far surpass the predictive qualities of
the $n$-gram. Allamanis et al.\@ state that they introduce a \enquote{\textelp{} model
that is to \textelp*{their} knowledge, the first that can propose neologisms
\textelp{}}~\cite{Allamanis2015Suggesting}.

Due to the sequential nature of text, the identifier naming problem is also particularly
well-suited for \acp{RNN} and \acp{LSTM}, although we do not include a study on the
latter. The paper by Gao et al.\@ focuses on an encoder-decoder \ac{RNN} model with the
promising addition of the attention and copy mechanisms~\cite{Gao2019IdentGen}. Instead of
reading the surrounding source code, documentation comments are used to train the model in
this study. The addition of attention makes it possible for the model to learn which words
are most important for each subtoken of the result. Neologisms are handled by copying,
which makes it possible for the model to select subtokens from the data directly, without
ever having learned them. Gao et al.\@ go on to show that an \ac{RNN} with both attention
and copying surpasses simpler \acp{RNN} and tf-idf, a widespread method of information
retrieval, in every metric that was measured.

% }}}

% Transformers, LLMs and Beyond {{{
\subsection{Transformers, \acp{LLM} and Beyond}
\label{ssec:LLMs-and-Beyond}

The most modern, and in turn, most sophisticated and performant \acp{LM} utilize
transformer models. Villmow et al.\@ propose such a transformer to discriminate between
identifiers that adhere to, or conversely, break naming
conventions~\cite{Villmow2023Violations}. Both a generative and a discriminative model was
trained: The generative model tries to fill in gaps in presented source code, while the
discriminative model uses a \enquote{weak \ac{AI}} to generate naming suggestions which
are then classified. The dataset used to train their \ac{LLM} was manually annotated and
contains over 6000 data points, which are freely available for future research. The model
that performed the best, dubbed \textsc{CodeDoctor} by the authors, was the generative
model with 247 million parameters. For comparison, the \textsc{GraphCodeBERT}
model~\cite{Guo2020GraphCodeBERT} based on BERT which was mentioned in
section~\ref{sec:Survey}, ranked third and contains 125 million parameters.

Another model based on BERT, and by far the most complex in terms of scope and execution
of the papers presented thus far, is the one presented by Ju et
al.\@~\cite{Ju2023XLangBinding}. This study delves into the scarcely researched topic of
tracking identifiers across programming languages and projects -- in fact, this is the
only paper in our survey to attempt this feat. They achieved cross-language identifier
binding by string-matching, and then passing the potential match through stacked
discriminators along with a pre-trained BERT model to apply contextual and framework
knowledge. Interestingly, this was also the only paper to compare their results with the
extremely popular IntelliJ IDEA: Whereas the IDE could correctly classify 564 unique and
1203 duplicate identifier pairs, the novel model could respectively detect 3361 and 8454
pairs.

% }}}

% }}}

% Discussion and Future Work {{{
% ~~~~~~~~~~~~~~~~~~~~ Discussion and Future Work ~~~~~~~~~~~~~~~~~~~~
\section{Discussion and Future Work}
\label{sec:Discussion-and-Future-Work}

Our survey shows that a lot of ongoing research in this field looks promising, and could
pose massive benefits for quality, maintainability, and validity of programs. Identifier
quality assessment seems to be a rather niche, but surprisingly deep topic. It is not hard
to imagine the potential for continuous integration pipelines to flag potentially
inappropriate identifier names, or for IDEs to support fully cross-language refactoring
tools. Such applications are still scarce, but do exist, as seen in the rename analysis
tool by Peruma et al.\@, which uses source code and commit messages in an attempt to
explain the decision making process behind name changes~\cite{Peruma2019}.

One subfield with increasing importance appears to be the splitting of identifiers into
subtokens that capture the meaning \emph{intended} by the developer with accuracy. Besides
the papers we have already presented~\cite{Gao2019IdentGen, Shi2022Splitting,
Allamanis2015Suggesting} there are clearly still improvements to be made, e.g.\@ Liu et
al.\@~\cite{Liu2021CHIS}, making this a potentially fruitful future research topic.

We acknowledge the potential of having missed some publications, as this survey was far
from extensive or rigorous. Furthermore, there is a possibility that literature on tools
already utilizing this research or on dedicated hobby projects has simply not been
published or reviewed through reputable outlets. We propose that further research is
needed to get a full grasp on how extensive tooling for this application of \acp{LM} is at
this point in time.

Perhaps this is the reason we were not able to find any publications involving the recent
\acp{GPT} with billions of parameters developed mostly by OpenAI. There seems to be a big
gap in understanding between the \enquote{small \acp{LLM}} and the incomprehensibly
complex models currently being trained. The efficacy of OpenAI's
ChatGPT\footnote{https://chat.openai.com} has recently come under more scrutiny, and our
conjecture is that targeted approaches will remain more effective for this use
case~\cite{Wu2023ChatGPT}.

% }}}

\enlargethispage{-18mm}

% Conclusion {{{
% ~~~~~~~~~~~~~~~~~~~~ Conclusion ~~~~~~~~~~~~~~~~~~~~
\section{Conclusion}
\label{sec:Conclusion}

This small-scale survey was compiled by querying two publishers, for which 34 results were
found to be relevant. We analyzed 7 of these papers in more detail, and presented a brief
overview of the research landscape as it currently stands. We found that the majority of
proposed solutions to identifier quality assessment use \acp{DNN} like \acp{RNN} or
transformers, but that simpler solutions like decision trees are still adequate in some
circumstances. Table~\ref{tab:Overview-of-ML-Techniques-in-our-Survey} shows a breakdown
of these categories. In our conclusion, we discussed potential applications for this
research and future work.

\begin{table}[h]
    \centering
    \caption{Overview of \acs{ML} Techniques in our Survey}
    \label{tab:Overview-of-ML-Techniques-in-our-Survey}
    \begin{tabular}{rl} \toprule
        Technique Used & \#\hspace{0.6em} \\ \midrule
        \ac{NN}        & 2 \\
        \ac{DNN}       & 2 \\
        \ac{RNN}       & 4 \\
        Transformer    & 8 \\[1.25ex]
        Other          & 9 \\[1.25ex]
        N.A.\@         & 9 \\ \bottomrule
        % \ac{DNN}       & 1 \\
        % \ac{RNN}       & 1 \\
        % Transformer    & 2 \\
        % Other          & 3 \\ \bottomrule
    \end{tabular}
\end{table}

% }}}

% References {{{
% ~~~~~~~~~~~~~~~~~~~~ References ~~~~~~~~~~~~~~~~~~~~

% % WARNING: This is a temporary macro to show references before we cite them in the text.
% % TODO(Marcel): REMOVE after outline completion
% \nocite{*}

\bibliography{IEEEfull,literature}
% }}}

\end{document}
% }}}

% vim: foldmethod=marker
